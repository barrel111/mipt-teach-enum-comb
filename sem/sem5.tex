\documentclass[a5paper]{article}
% Uncomment the following line to allow the usage of graphics (.png, .jpg)
%\usepackage[pdftex]{graphicx}
% Comment the following line to NOT allow the usage of umlauts

\pagestyle{empty}
\usepackage[T2A]{fontenc}
\usepackage[utf8]{inputenc}
\usepackage[russian]{babel}
\usepackage{cmap}
\usepackage{amsthm}
\usepackage{amsmath}
\usepackage{units}
\usepackage{fancyhdr}
\usepackage{forloop}
\usepackage{amssymb}
\usepackage{url}
\usepackage{hyperref}
\usepackage{xcolor}
\usepackage[inline]{enumitem}
\usepackage{graphicx}
\usepackage{caption}
\usepackage{subcaption}
\usepackage{amscd}
\usepackage[ruled,vlined]{algorithm2e}

%------------- TIKZ --------------------
\definecolor{bblue}{rgb}{0.2, 0.4, 0.8}
\definecolor{bgreen}{rgb}{0.2, 0.6, 0.4}
\definecolor{bred}{rgb}{0.8, 0.4, 0.2}
\definecolor{bviolet}{rgb}{0.7, 0.2, 0.7}
\definecolor{blackred}{rgb}{0.6, 0.3, 0.3}
\definecolor{blackblue}{rgb}{0.3, 0.3, 0.6}

\usepackage{tikz}
\usetikzlibrary{arrows,trees,shapes,snakes,shapes.geometric}
\tikzset{
  treenode/.style = {align=center, inner sep=0pt, text centered,
    font=\sffamily},
  arn_nn/.style = {treenode, circle, bblue, draw=bblue, 
    fill=bblue!10,
    minimum width=0.5em, minimum height=0.5em
},% arbre rouge noir, noeud rouge
  arn_n/.style = {treenode, circle, bblue, draw=bblue, 
    text width=1.5em, very thick,
    fill=bblue!10},% arbre rouge noir, noeud rouge
  arn_g/.style = {treenode, circle, bgreen, draw=bgreen, 
    text width=1.5em, very thick,
    fill=bblue!10},% arbre rouge noir, noeud rouge
  arn_r/.style = {treenode, circle, bviolet, draw=bviolet, 
    text width=1.5em, very thick,
    fill=bviolet!10},% arbre rouge noir, noeud rouge
  arn_x/.style = {treenode, triangle, draw=black,
    minimum width=0.5em, minimum height=0.5em},% arbre rouge noir, nil
  triangle/.style = {treenode, bred, draw=bred, fill=bred!20, regular polygon, regular polygon
    sides=3, very thick, text width=1.5em }
}

\renewcommand{\thesection}{\arabic{section}}

\renewcommand{\headrulewidth}{0.4pt}
\renewcommand{\footrulewidth}{0.4pt}

\fancyhead[R]{\thepage}
%For multipage documents only!
%\fancyfoot[L]{page: \thepage}
\fancyhead[L]{Перечислительная комбинаторика}
\fancyfoot[C]{\topic}
\pagestyle{fancy}

\renewcommand{\baselinestretch}{1.0}
\renewcommand\normalsize{\sloppypar}

\setlength{\topmargin}{-0.5in}
\setlength{\textheight}{6.5in}
\setlength{\oddsidemargin}{-0.3in}
\setlength{\evensidemargin}{-0.3in}
\setlength{\textwidth}{4.5in}
\setlength{\parindent}{0ex}

\newcounter{problemset}
%Here you should set the total number of pages
\setcounter{totalpages}{1}

\def \topic {Семинар 5}

\def \Z {\mathbb Z}
\def \R {\mathbb R}
\def \P {\mathbb P}
\def \C {\mathbb C}
\def \vec {\boldsymbol}

\theoremstyle{definition}
\newtheorem{lemma}{Лемма}
\newtheorem{example}{Пример}
\newtheorem*{theorem}{Теорема}
\newtheorem*{definition}{Определение}

\usepackage{titlesec}

\makeatletter
\renewcommand{\section}{\@startsection
{section}%                   % the name
{1}%                         % the level
{\z@}%                       % the indent / 0mm
{-\baselineskip}%            % the before skip / -3.5ex \@plus -1ex \@minus 
%%-.2ex
{0.5\baselineskip}%          % the after skip / 2.3ex \@plus .2ex
{\centering\large\scshape}} % the style

\renewcommand{\subsection}{\@startsection
{subsection}%                % the name
{1}%                         % the level
{\z@}%                       % the indent / 0mm
{-\baselineskip}%            % the before skip / -3.5ex \@plus -1ex \@minus 
%%-.2ex
{0.5\baselineskip}%          % the after skip / 2.3ex \@plus .2ex
{\centering\large\scshape}} % the style
\makeatother


\begin{document}

\begin{center}

\newcommand{\HRule}{\rule{\linewidth}{0.5mm}}
\HRule \\[0.2cm]
{ \Large \bfseries \topic} %\\[0.2cm]
\HRule

\end{center}

\textsc{Ключевые слова: рекурсивные 
структуры, генератор Больцмана, формула обращения Лагранжа, обобщённые деревья Кэли.}

\section{Разминка и маркировка}
\begin{example}
    Рассмотрим класс корневых непомеченных деревьев, где потомки каждой вершины
упорядочены, другими словами, дерево это вершина и последовательность деревьев
(возможно, нулевой длины). Таким образом, дерево рекурсивно задаётся от корня к
поддеревьям, которые в свою очередь тоже являются деревьями.
Найдите математическое ожидание количества \textit{листьев} в случайно выбранном
дереве размера \( n \).
\end{example}
\begin{figure}[hbt]
\centering
\begin{tikzpicture}[>=stealth',level/.style={sibling distance = 1.5cm/#1,
  level distance = 1.5cm, thick}] 
\draw
node[triangle]{\( T \)}
++(1.1,0)
node{\( \boldsymbol = \)}
++(1,0.5)
node(up)[arn_n]{\( \bullet \)}
    child{
        node[triangle]{\( T \)}
    }
    child{
        node[triangle]{\( T \)}
    }
++(0,-1.5)
node{\( \cdots \)}
;
\end{tikzpicture}
\caption{\label{fig:tree:specification}
Иллюстрация комбинаторного определения деревьев.}
\end{figure}
Обыкновенная производящая функция \( T(x) \) для деревьев (Каталана)
удовлетворяет уравнению
\begin{equation}
    T(x) = x \cdot \dfrac{1}{1- T(x)}
    \enspace .
\end{equation}
Чтобы ответить на вопрос задачи, недостаточно просто решить это уравнение, но мы
на всякий случай это сделаем. Я хочу обратить внимание на то, что пристальное
вглядывание в это уравнение позволяет нам убедиться, что оно имеет единственное
решение в классе формальных степенных рядов. Действительно, мы видим, что \( T(0) =
0 \), то есть свободный член равен нулю. Поэтому разложение \( \dfrac{1}{1 -
  T(x)} \) начинается с единицы, и коэффициент при \( x \) в правой части равен
\( 1 \). Имея эту информацию, можно подставить \( T(x) = x + Q(x) \), получая
\begin{equation}
   T(x) = \dfrac{x}{1 - T(x)} = \dfrac{x}{1 - x - Q(x)} = 
    x (1 + (x + Q(x)) + \ldots) = x + x^2 + \ldots
\end{equation}
Таким образом, за второй шаг подстановки мы можем понять, что коэффициент при \(
x^2 \) равен \( 1 \). Кстати, если вы внимательно читали предыдущие семинары, то
знаете, что можно воспользоваться итерацией Ньютона-Рафсона, сразу получая за
один шаг \( 2^n \) коэффициентов.

Уравнение, конечно же, является квадратным относительно \( T(x) \), и все мы умеем решать квадратные
уравнения, поэтому я сразу выписываю то решение, которое является формальным
степенным рядом (второе не является)
\begin{equation}
    T(x) = \dfrac{1 - \sqrt{1 - 4x}}{2} \enspace .
\end{equation}
\def\uu{ {\color{blackred}u} }
Затем рассмотрим производящую функцию от двух переменных \( T(z, \uu) \), 
где переменная \( \uu \) маркирует листья дерева, то есть коэффицент
\( [z^n \uu^k] T(z, \uu) \) равен количеству деревьев на \( n \) вершинах, у
которых имеется \( k \) листьев. Как построить такую функцию? Нужно лишь учесть
листовые вершины в конструкции дерева.

\begin{figure}[hbt]
\centering
\begin{tikzpicture}[>=stealth',level/.style={sibling distance = 1.5cm/#1,
  level distance = 1.5cm, thick}] 
\draw
node[triangle]{\( T \)}
++(1.1,0.2)
node{\( \boldsymbol = \)}
++(1,0)
node[arn_g]{ \( \uu \) }
++(1,0)
node{\( \boldsymbol + \)}
++(1,0.5)
node(up)[arn_n]{\( \bullet \)}
    child{
        node[triangle]{\( T \)}
    }
    child{
        node[triangle]{\( T \)}
    }
++(0,-1.5)
node{\( \cdots \)}
++(0,-0.65)
node{\(\underbrace{\phantom{\qquad \qquad \qquad \qquad}}_{\geq 1} \)}
;
\end{tikzpicture}
\caption{\label{fig:tree:marked}
Деревья с помеченными листьями.}
\end{figure}
Получаем уравнение на производящую функцию
\begin{equation}
    T(z, \uu) = z \uu + \dfrac{z T(z, \uu)}{1 - T(z, \uu)}
    \enspace ,
\end{equation}
и это уравнение тоже квадратное, и снова легко решается:
\begin{equation}
    T(z, \uu) = \dfrac{z\uu - z + 1 - \sqrt{(z\uu-z+1)^2-4z\uu}}{2}
    \enspace .
\end{equation}
Можно заметить, что подстановка \( \uu = 1 \) возвращает нас к непомеченным
деревьям, значит, мы всё делаем правильно.

При этом та самая производящая функция моментов, которая возникает в теории
вероятности, здесь поможет нам посчитать матожидание. В действительности, 
если обозначить \( L_n \) случайную величину, которая равна количеству листьев в
случайно выбранном дереве размера \( n \), то для неё выполнено
\begin{equation}
    \mathbb E u^{L_n} = \dfrac{[z^n] T(z, \uu)}{[z^n]T(z,1)} 
    \enspace ,
\end{equation}
Значит, матожидание можно найти, продифференцировав обе части:
\begin{equation}
    \mathbb E L_n = \left. \dfrac{d}{d \uu} \mathbb E u^{L_n} \right|_{\uu = 1}
    = \dfrac
        {\left. [z^n] \dfrac{d}{d \uu} T(z, \uu)\right|_{\uu = 1}}
        {[z^n] T(z, 1)}
    \enspace .
\end{equation}
Дифференцируем числитель и подставляем \( \uu = 1 \). Получаем:
\begin{equation}
    T'_{\uu}(z, 1) = z + \dfrac{z}{2 \sqrt{1 - 4z}} \enspace .
\end{equation}
Отношение числителя и знаменателя можно выразить с помощью биномиальной теоремы:
\begin{multline*}
    \dfrac
        { [z^n] \tfrac12 z(1-4z)^{-1/2} }
        { [z^n] -\tfrac12 (1-4z)^{1/2} }
    = 
    \dfrac
        { (-4)^{n-1} {-1/2 \choose n-1} }
        { -(-4)^n {1/2 \choose n} }
    \\=
    \dfrac{n! (-\tfrac12)(-\tfrac12-1) \ldots (-\tfrac12 - (n-1) + 1)}
    {4(n-1)! (\tfrac12)(\tfrac12 - 1) \ldots (\tfrac12 - n + 1)}
    = \dfrac{n}{8}
    \enspace .
\end{multline*}
Таким образом, при \( n \geq 2 \) математическое ожидание доли листьев в случайно
выбранном дереве составляет в точности \( \dfrac18 \).

\section{Чтобы понять рекурсию, надо сперва понять рекурсию}
Теперь обратимся к компьютерной генерации случайных деревьев.
Естественно, после написания соответствующего генератора доказанное нами
утверждение про \( \dfrac n8 \) можно будет затем проверить экспериментально.
Для этих целей очень часто используют язык
программирования \texttt{Haskell}.

Генератор Больцмана, про который я собираюсь рассказать, к сожалению, не сможет нам предоставить
для заданного фиксированного \( n \) случайные деревья, имеющие строго заданный
размер, но зато он может гарантировать, что внутри каждого размера распределение
деревьев будет равномерным, а ещё что средний размер будет приблизительно равен
\( n \).

Мы хотим выбрать некое значение параметра \( x \geq 0 \) и генерировать объекты
с вероятностью
\begin{equation}
    \mathbb P( |\text{\textsc{tree}}| = n) = \dfrac{t_n x^n}{T(x)} \enspace ,
\end{equation}
где \( t_n \) это коэффициент при \( x^n \) в функции \( T(x) \).
\begin{example}
    Пусть мы ещё и не умеем генерировать такую случайную величину, но она корректно
определена, так что мы можем попробовать найти её математическое ожидание и
дисперсию.
    
    Заметим, что производящая функция моментов этой случайной величины имеет вид
\begin{equation}
    \mathbb E \uu^{ |\text{\textsc{tree}}|} =
    \sum_n \mathbb P_x ( |\text{\textsc{tree}}| = n) \uu^n
    = \dfrac{T(\uu x)}{T(x)} \enspace.
\end{equation}
Матожидание и дисперсия получаются дифференцированием:
\begin{equation}
    \mathbb E_x ( | \text{\textsc{tree}} | ) = \left.
        \dfrac{d}{d \uu} \dfrac{T(\uu x)}{T(x)} 
    \right|_{\uu = 1}
    = x \dfrac{T'(x)}{T(x)}
    \enspace , \ 
    \mathbb E_x |\text{\textsc{tree}}|^2 = 
    \dfrac
        {x^2 T''(x) + x T'(x)}{T(x)}
    \enspace ,
\end{equation}
и можно проверить, что 
\begin{equation}
    \mathrm{var}_x |\text{\textsc{tree}}| = 
    x \dfrac{d}{dx} \mathbb E_x  |\text{\textsc{tree}}| 
    \geq 0 \enspace ,
\end{equation}
откуда, в частности, следует, что функция \( \phi(x) = x \dfrac{T'(x)}{T(x)} \)
всегда является неубывающей, а для невырожденной величины ещё и возрастающей на
своей области определения. Более того, для любого \( n \) можно найти такой \(
x(n) \) в круге сходимости, что математическое ожидание построенной случайной
величины будет равно \( n \), при этом дисперсия обычно тоже будет иметь порядок
\( n \).
\end{example}
\subsection{Алгоритм}
Алгоритм принимает на вход \textit{спецификацию класса объектов} и возвращает
случайный объект из нужного класса. Любую спецификацию можно задать рекурсивно с
помощью элементарных операций типа объединения, декартова произведения,
последовательности (иногда дифференцирования), поэтому мы рассмотрим отдельно
первые три случая.

\begin{algorithm}[hbt]
\KwData{\( \mathcal C = \mathcal A + \mathcal B \)}
\KwResult{ Случайный объект из класса \( \mathcal C \)}
\Begin{
\( p = \dfrac{A(x)}{A(x) + B(x)} \)\;
Генерируем бернуллиевскую случайную величину \( Be(p) \)\;
\eIf{ \( Be(p) \) }
    { выдаём объект класса \( \mathcal A \) }
    { выдаём объект класса \( \mathcal B \) }
}
\caption{Объединение классов}
\end{algorithm}

\begin{algorithm}[hbt]
\KwData{\( \mathcal C = \mathcal A \times \mathcal B \)}
\KwResult{ Случайный объект из класса \( \mathcal C \)}
\Begin{
Генерируем независимо \( (a, b) \in (\mathcal A \times \mathcal B) \)\;
}
\caption{Декартово произведение классов}
\end{algorithm}

Для оператора \text{\textsc{seq}} существует два эквивалентных способа генерации
последовательности. Напомню, что последовательность всегда имеет конечную, пусть
и случайную, длину. Заметим, что класс последовательностей \( \mathcal C =
\text{\textsc{seq}}(\mathcal A) \) удовлетворяет
рекурсивному уравнению
\begin{equation}
    \mathcal C = \boldsymbol 1 + \mathcal A \times \mathcal C \enspace ,
\end{equation}
где \( \boldsymbol 1 \) это пустая последовательность. 

\begin{algorithm}[hbt]
\KwData{\( \mathcal C = \text{\textsc{seq}} ( \mathcal A ) \)}
\KwResult{ Случайный объект из класса \( \mathcal C \)}
\Begin{
\( p = A(x) \)\;
Генерируем бернуллиевскую случайную величину \( Be(p) \)\;
\eIf{ \( Be(p) \) }
    { выдаём кортеж \( (a, c) \in ( \mathcal A \times \mathcal C ) \), где \( c
\) строится рекурсивно }
    { выдаём \( \boldsymbol 1 \)~--- пустую последовательность }
}
\caption{Рекурсивный алгоритм для \textsc{seq}.}
\end{algorithm}

\begin{algorithm}[hbt]
\KwData{\( \mathcal C = \text{\textsc{seq}} ( \mathcal A ) \)}
\KwResult{ Случайный объект из класса \( \mathcal C \)}
\Begin{
\( p = A(x) \)\;
Генерируем геометрическую случайную величину \( k = Geom(p) \)\;
Возвращаем кортеж \( (a_1, a_2, \ldots, a_k) \)\;
}
\caption{Геометрический алгоритм для \textsc{seq}.}
\end{algorithm}
\begin{theorem}
    Все четыре алгоритма работают, то есть если в распоряжении имеются
корректные генераторы Больцмана для дочерних классов, то полученный генератор
для комбинаторной операции тоже будет иметь правильное распределение, а объекты
одного и того же размера будут равновероятны.
\end{theorem}
\begin{proof}
    См. \cite{boltzmann}.
\end{proof}
\begin{example}
    Применим этот алгоритм к генерации деревьев.
    Шаг алгоритма состоит в том, чтобы выдать кортеж, состоящий из вершины
дерева, и последовательности поддеревьев длины \( Geom(T(x)) \), где \( T(x) =
\tfrac{1 - \sqrt{1 - 4x}}{2} \).
\end{example}
\begin{example}[Бинарные деревья]
    Бинарные деревья задаются с помощью спецификации
\begin{equation}
    \mathcal B = \bullet + ( \mathcal B \times \bullet \times \mathcal B)
\end{equation}
Следовательно, алгоритм работает так: находит число \( p = \dfrac{x}{B(x)} \),
где \( B(x) = \tfrac{1 - \sqrt{1 - 4x^2}}{2x} \), генерирует бернуллиевскую
величину \( Be(p) \). В случае успеха возвращает единственную вершину, в
противном случае бинарное дерево с вершиной и двумя потомками, сгенерированными
рекурсивно.
\end{example}

\section{Задачи}
\begin{enumerate}
\item(1 очко) Покажите, что факториальный момент порядка \( k \) можно
вычислить, дифференцируя функцию моментов \( k \) раз, то есть
\begin{equation}
    \mathbb E (X(X - 1) \ldots (X - k + 1))
    = \left.\dfrac{d^k}{du^{k}} \mathbb E u^{X}
\right|_{u=1}
\end{equation}
\item(2 очка) Сгенерируйте деревья Моцкина, заданные спецификацией
\begin{equation}
    \mathcal M = z \times( \boldsymbol 1 + \mathcal M + \mathcal M^2) \enspace ,
\end{equation}
то есть деревья, где у каждой вершины не более 2 потомков. Для любого заданного
числа \( n \) (матожидания размера дерева) нужно уметь находить число \( x
\)~--- параметр генератора Больцмана, и генерировать случайное дерево большого
размера (от 10 тысяч до миллиона).
\item(3 очка) Дополнительно к предыдущей задаче, изучите какие-нибудь интересные
статистики построенных деревьев: найдите среднее значение числа листьев, среднюю
высоту дерева, количество унарных вершин, количество бинарных вершин. Найдите,
как эти параметры зависят от числа \( n \) (линейно?). Как дисперсия этих
параметров зависит от \( n \)?

\item(2 очка) Разберитесь, как работает генератор Больцмана для помеченных
классов объектов, а также как устроена операция \textsc{set},
\textsc{cyc}, операция дифференцирования и композиции.

\item(3 очка) Воспользуйтесь тем, что 
\begin{equation}
    [z^n] \dfrac{1}{1 - z} \log \dfrac{1}{1-z} = 1 + \dfrac12 + \ldots +
\dfrac1n \enspace ,
\end{equation}
и докажите, что математическое ожидание числа циклов в случайной перестановке
имеет порядок \( \log n \), а дисперсия этой величины имеет вид
\begin{equation}
    \sigma^2 = \sum_{k=1}^n \dfrac 1k - \sum_{k=1}^n \dfrac{1}{k^2}
\end{equation}
\item(2+1 очкo) Рассмотрите класс \textit{разбиений} на положительные слагаемые,
то есть  \( \text{\textsc{seq}}(\text{\textsc{seq}}_{\geq 1}(\bullet)) \),
пометив отдельной переменной слагаемые, равные 1, и отдельной переменной
слагаемые, равные 2.

(i) Найдите производящую функцию от трёх переменных для
полученного класса.

(ii) Найдите среднее количество слагаемых в разложении.
\item(3 очка) Докажите, что математическое ожидание \textit{степени вершины} в
случайном дереве Каталана размера \( n \) равно
\begin{equation}
    \mathbb E_n (\text{deg}) = 3 \dfrac{n-1}{n+1} \enspace ,
\end{equation}
в частности эта величина асимптотически равна \( 3 \).
\item(1 + 2 очка) Сотне студентов, приговорённых к последней пересдаче по дискретному анализу, был дан
последний шанс. Студентов по очереди вводят в комнату, где располагаются 100
ящиков, помеченных числами от 1 до 100, и в которых лежат в каком-то порядке
номера билетов. Каждый студент может перетянуть билет не более 49 раз, то есть
всего попробовать не более 50 билетов. Каждый студент выучил ровно один билет,
поэтому если он не угадает, то его ждёт пересдача.
Общаться при этом с другими студентами запрещено. Если хотя бы один не справляются, всех отправляют в армию.
Иннокентий пессемистично утверждает, что шансы на успех имеют порядок \(1 / 2^{100}
 = 8 \cdot 10^{-31}\), тогда как Эдуард, более сведущий в комбинаторике,
утверждает, что существует стратегия с вероятностью сдачи более 30 процентов.
Кто из студентов прав?

Альтернативный вариант задачи: количество студентов 150, и можно вытягивать
билет не более 50 раз. Докажите, что вероятность не попасть в армию не менее 
\( 1 - \log 3 + (\log 3/2)^2 \approx 6.5\%  \)
\end{enumerate}
\footnotesize
\bibliographystyle{plain}
\bibliography{biblio}
    
\end{document}
